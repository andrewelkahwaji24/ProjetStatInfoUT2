% Options for packages loaded elsewhere
\PassOptionsToPackage{unicode}{hyperref}
\PassOptionsToPackage{hyphens}{url}
%
\documentclass[
]{article}
\usepackage{amsmath,amssymb}
\usepackage{iftex}
\ifPDFTeX
  \usepackage[T1]{fontenc}
  \usepackage[utf8]{inputenc}
  \usepackage{textcomp} % provide euro and other symbols
\else % if luatex or xetex
  \usepackage{unicode-math} % this also loads fontspec
  \defaultfontfeatures{Scale=MatchLowercase}
  \defaultfontfeatures[\rmfamily]{Ligatures=TeX,Scale=1}
\fi
\usepackage{lmodern}
\ifPDFTeX\else
  % xetex/luatex font selection
\fi
% Use upquote if available, for straight quotes in verbatim environments
\IfFileExists{upquote.sty}{\usepackage{upquote}}{}
\IfFileExists{microtype.sty}{% use microtype if available
  \usepackage[]{microtype}
  \UseMicrotypeSet[protrusion]{basicmath} % disable protrusion for tt fonts
}{}
\makeatletter
\@ifundefined{KOMAClassName}{% if non-KOMA class
  \IfFileExists{parskip.sty}{%
    \usepackage{parskip}
  }{% else
    \setlength{\parindent}{0pt}
    \setlength{\parskip}{6pt plus 2pt minus 1pt}}
}{% if KOMA class
  \KOMAoptions{parskip=half}}
\makeatother
\usepackage{xcolor}
\usepackage[margin=1in]{geometry}
\usepackage{graphicx}
\makeatletter
\def\maxwidth{\ifdim\Gin@nat@width>\linewidth\linewidth\else\Gin@nat@width\fi}
\def\maxheight{\ifdim\Gin@nat@height>\textheight\textheight\else\Gin@nat@height\fi}
\makeatother
% Scale images if necessary, so that they will not overflow the page
% margins by default, and it is still possible to overwrite the defaults
% using explicit options in \includegraphics[width, height, ...]{}
\setkeys{Gin}{width=\maxwidth,height=\maxheight,keepaspectratio}
% Set default figure placement to htbp
\makeatletter
\def\fps@figure{htbp}
\makeatother
\setlength{\emergencystretch}{3em} % prevent overfull lines
\providecommand{\tightlist}{%
  \setlength{\itemsep}{0pt}\setlength{\parskip}{0pt}}
\setcounter{secnumdepth}{-\maxdimen} % remove section numbering
\ifLuaTeX
  \usepackage{selnolig}  % disable illegal ligatures
\fi
\usepackage{bookmark}
\IfFileExists{xurl.sty}{\usepackage{xurl}}{} % add URL line breaks if available
\urlstyle{same}
\hypersetup{
  pdftitle={MI0A403T - Statistique inférentielle / Projet informatique-statistique},
  pdfauthor={Andrew El Kahwaji, Wael Aboulkacem, Hans Kanen Soobbooroyen},
  hidelinks,
  pdfcreator={LaTeX via pandoc}}

\title{MI0A403T - Statistique inférentielle / Projet
informatique-statistique}
\author{Andrew El Kahwaji, Wael Aboulkacem, Hans Kanen Soobbooroyen}
\date{15/04/2025}

\begin{document}
\maketitle

{
\setcounter{tocdepth}{3}
\tableofcontents
}
\subsection{1. Objectif du Projet}\label{objectif-du-projet}

Ce projet qui est liee a l'UE Statistique Inferentielle / Projet
Stat-Info qui vise à mieux comprendre les raisons et cause des incendies
en analysant des données statistiques. L'idée principale est de voir
comment différentes variables influencent l'étendue des incendies, en
utilisant des outils statistiques et informatiques.

\subsection{2. Partie Informatique}\label{partie-informatique}

La partie Informatique de notre Projet consiste à effectuer les
démarches suivantes :

\begin{enumerate}
\def\labelenumi{\arabic{enumi}.}
\tightlist
\item
  Création de la Base de données
\item
  Création de la connexion entre la Base de données et notre code source
\item
  Établissement des Tables dans la Base de Données
\item
  Insertion de données dans les tables
\item
  Présenter les informations des tables dans la console
\item
  Exportation des données dans les tables dans des fichiers CSV
\end{enumerate}

\subsubsection{2.1 Bibliotheques
Utilisees}\label{bibliotheques-utilisees}

Dans notre Partie Informatique on a utilisee le Language de
Programmation Python de plus pour pouvoir effectuer la manipulation des
doneees de la maniere optimale on a utilisee les bibliotheques
necessaires:

\begin{enumerate}
\def\labelenumi{\arabic{enumi}.}
\tightlist
\item
  sqlite3 SQLite3 est une bibliothèque peu aisée qui facilite
  l'incorporation d'une base de données au sein d'une application, sans
  nécessiter l'utilisation d'un serveur séparé. Elle offre la
  possibilité de stocker et de gérer des données grâce aux requêtes SQL,
  ce qui la rend pratique pour des projets nécessitant une base de
  données locale. SQLite3 est parfaitement adaptée aux applications
  simples, car elle offre une gestion aisée des données, que ce soit
  pour les ajouts, les changements ou les suppressions, tout en restant
  performante et peu gourmande en ressources.
\item
  csv
\end{enumerate}

```python import sqlite3

\section{Connexion à la base de données (elle sera créée si elle
n'existe
pas)}\label{connexion-uxe0-la-base-de-donnuxe9es-elle-sera-cruxe9uxe9e-si-elle-nexiste-pas}

conn = sqlite3.connect(`exemple.db')

\section{Création d'un curseur pour exécuter des
requêtes}\label{cruxe9ation-dun-curseur-pour-exuxe9cuter-des-requuxeates}

cur = conn.cursor()

\section{Création d'une table}\label{cruxe9ation-dune-table}

cur.execute('\,`'CREATE TABLE IF NOT EXISTS utilisateurs (id INTEGER
PRIMARY KEY, nom TEXT, age INTEGER)'\,'\,')

\section{Insertion de données}\label{insertion-de-donnuxe9es}

cur.execute(``INSERT INTO utilisateurs (nom, age) VALUES (`Alice',
30)'')

\section{Sauvegarde des modifications et fermeture de la
connexion}\label{sauvegarde-des-modifications-et-fermeture-de-la-connexion}

conn.commit() conn.close()

\subsubsection{2.2 Création de la base de
données}\label{cruxe9ation-de-la-base-de-donnuxe9es}

Les données sont collectées à partir de sources disponibles et intégrées
dans une base de données structurée pour une analyse statistique
approfondie.

\end{document}
