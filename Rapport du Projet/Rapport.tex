% Options for packages loaded elsewhere
\PassOptionsToPackage{unicode}{hyperref}
\PassOptionsToPackage{hyphens}{url}
%
\documentclass[
]{article}
\usepackage{amsmath,amssymb}
\usepackage{iftex}
\ifPDFTeX
  \usepackage[T1]{fontenc}
  \usepackage[utf8]{inputenc}
  \usepackage{textcomp} % provide euro and other symbols
\else % if luatex or xetex
  \usepackage{unicode-math} % this also loads fontspec
  \defaultfontfeatures{Scale=MatchLowercase}
  \defaultfontfeatures[\rmfamily]{Ligatures=TeX,Scale=1}
\fi
\usepackage{lmodern}
\ifPDFTeX\else
  % xetex/luatex font selection
\fi
% Use upquote if available, for straight quotes in verbatim environments
\IfFileExists{upquote.sty}{\usepackage{upquote}}{}
\IfFileExists{microtype.sty}{% use microtype if available
  \usepackage[]{microtype}
  \UseMicrotypeSet[protrusion]{basicmath} % disable protrusion for tt fonts
}{}
\makeatletter
\@ifundefined{KOMAClassName}{% if non-KOMA class
  \IfFileExists{parskip.sty}{%
    \usepackage{parskip}
  }{% else
    \setlength{\parindent}{0pt}
    \setlength{\parskip}{6pt plus 2pt minus 1pt}}
}{% if KOMA class
  \KOMAoptions{parskip=half}}
\makeatother
\usepackage{xcolor}
\usepackage[margin=1in]{geometry}
\usepackage{graphicx}
\makeatletter
\def\maxwidth{\ifdim\Gin@nat@width>\linewidth\linewidth\else\Gin@nat@width\fi}
\def\maxheight{\ifdim\Gin@nat@height>\textheight\textheight\else\Gin@nat@height\fi}
\makeatother
% Scale images if necessary, so that they will not overflow the page
% margins by default, and it is still possible to overwrite the defaults
% using explicit options in \includegraphics[width, height, ...]{}
\setkeys{Gin}{width=\maxwidth,height=\maxheight,keepaspectratio}
% Set default figure placement to htbp
\makeatletter
\def\fps@figure{htbp}
\makeatother
\setlength{\emergencystretch}{3em} % prevent overfull lines
\providecommand{\tightlist}{%
  \setlength{\itemsep}{0pt}\setlength{\parskip}{0pt}}
\setcounter{secnumdepth}{-\maxdimen} % remove section numbering
\ifLuaTeX
  \usepackage{selnolig}  % disable illegal ligatures
\fi
\usepackage{bookmark}
\IfFileExists{xurl.sty}{\usepackage{xurl}}{} % add URL line breaks if available
\urlstyle{same}
\hypersetup{
  pdftitle={MI0A403T - Statistique inférentielle / Projet informatique-statistique},
  pdfauthor={Andrew El Kahwaji, Wael Aboulkacem, Hans Kanen Soobbooroyen},
  hidelinks,
  pdfcreator={LaTeX via pandoc}}

\title{MI0A403T - Statistique inférentielle / Projet
informatique-statistique}
\author{Andrew El Kahwaji, Wael Aboulkacem, Hans Kanen Soobbooroyen}
\date{09/05/2025}

\begin{document}
\maketitle

{
\setcounter{tocdepth}{3}
\tableofcontents
}
\subsection{1. Objectif du Projet}\label{objectif-du-projet}

Ce projet qui est liée a l'UE Statistique Inferentielle / Projet
Stat-Info qui vise à mieux comprendre les raisons et cause des incendies
en analysant des données statistiques. L'idée principale est de voir
comment différentes variables influencent l'étendue des incendies, en
utilisant des outils statistiques et informatiques.

Nous avons structuré notre rapport en trois sections principales : la
Partie Informative, qui servira à présenter des informations générales
afin d'aider le lecteur à comprendre notre projet avant de plonger dans
les détails et les spécificités des sections Informatique et
Statistique. La section Informatique va examiner minutieusement les
techniques que nous avons mises en œuvre dans notre projet, en précisant
toutes les informations indispensables. Et pour finir, la section
Statistique qui nous aidera à détailler toutes les études que nous avons
réalisées avec les diagrammes appropriés, l'interprétation et la
solution de nos enjeux.

Finalement, nous avons constitué une section supplémentaire qui nous
sert à énumérer tous les sites internet que nous avons consultés pour la
rédaction et l'élaboration de notre projet. Il est à noter que les
références sont présentées au format APA !

\subsection{2. Partie Informative}\label{partie-informative}

Dans cette partie, nous allons considérer des données informatives avant
de passer à la description de notre section Informatique et Statistique.

\subsubsection{2.1 C'est quoi un incendie}\label{cest-quoi-un-incendie}

L'incendie est un phénomène de combustion incontrôlée dans le temps et
l'espace, dont la principale caractéristique est sa capacité à se
propager rapidement.

Pour qu'une combustion puisse se produire, trois éléments habituellement
réunis dans le « triangle du feu » sont indispensables : un matériau
combustible, un agent comburant et une source d'énergie d'activation.

\subsubsection{2.2 Causes des Incendies}\label{causes-des-incendies}

Les raisons des incendies sont multiples, cependant, une grande majorité
d'entre eux provient d'une action humaine.Comme la négligence,
malveillance, préparation insuffisante aux catastrophes naturelles comme
les séismes, les tsunamis

\subsubsection{2.3 Consequences des
Incendies}\label{consequences-des-incendies}

Les effets des incendies sont nombreux et graves. Elles ont des
conséquences sur l'homme (asphyxie, intoxication due aux fumées,
brûlures sévères), sur les entreprises (diminution de la production,
dégâts matériels, licenciements) et sur l'environnement (contamination
de l'air et de l'eau, ravage du paysage). Les principales causes de
décès liés aux incendies sont l'intoxication par le monoxyde de carbone
et la diminution de l'oxygène, plutôt que les flammes elles-mêmes. Pour
plus d'informations, veuillez consulter l'article intégral ici.

\subsection{3. Partie Informatique}\label{partie-informatique}

La partie Informatique de notre Projet consiste à effectuer les
démarches suivantes :

\begin{enumerate}
\def\labelenumi{\arabic{enumi}.}
\tightlist
\item
  Création de la Base de données
\item
  Création de la connexion entre la Base de données et notre code source
\item
  Établissement des Tables dans la Base de Données
\item
  Insertion de données dans les tables
\item
  Présenter les informations des tables dans la console
\item
  Exportation des données dans les tables dans des fichiers CSV
\end{enumerate}

\subsubsection{2.1 Definition de quelques terme
Informatiques}\label{definition-de-quelques-terme-informatiques}

Avant d'initier notre compte-rendu en détaillant les phases et
procédures que nous avons mises en place pour l'administration optimale
de la section informatique de notre projet, nous allons définir quelques
notions qui offriront un socle solide au lecteur.

\begin{enumerate}
\def\labelenumi{\arabic{enumi}.}
\item
  Une \textbf{Base de Données} regroupe un ensemble d'informatins qui
  est organise pour etre accessible , gere et mise a jour facilement par
  ses utilisateurs
\item
  Une \textbf{Base de Données Relationnelle}.Il s'agit d'un type de base
  de données qui se distingue des autres par sa capacité à établir des
  liens entre diverses données.
\item
  Un \textbf{INNER JOIN} est un type de jointure en SQL (Structured
  Query Language) qui autorise la fusion de lignes issues de deux tables
  basées sur un critère déterminé. L'idée est de ne conserver que les
  lignes qui ont une correspondance dans les deux tables. De plus, si
  une ligne d'une table n'a pas de correspondance dans l'autre table,
  elle ne sera pas intégrée au résultat.
\item
  Un \textbf{Cast()} est une fonction SQL appelée CAST nous donne la
  possibilité de transformer un type de données en un autre.
\item
  Une \textbf{Requête SQL} s'apparente à une question formulée à la base
  de données afin d'extraire des informations de celle-ci.
\item
  La fonction SQL \textbf{COUNT()} est une commande qui nous donne la
  possibilité de déterminer le nombre de lignes dans un ensemble de
  résultats. Elle est fréquemment utilisée en conjonction avec la clause
  GROUP BY pour recenser le nombre d'occurrences d'une valeur
  spécifique.
\item
  La fonction \textbf{SUBSTR()} nous donne la possibilité d'extraire une
  portion d'une chaîne de caractères. Elle est souvent employée dans le
  domaine du développement pour l'édition de texte dans une ou plusieurs
  bases de données.
\end{enumerate}

\subsubsection{2.2 Bibliotheques
Utilisees}\label{bibliotheques-utilisees}

Dans notre Partie Informatique on a utilisee le Language de
Programmation Python de plus pour pouvoir effectuer la manipulation des
doneees de la maniere optimale on a utilisee les bibliothèques
necessaires:

\begin{enumerate}
\def\labelenumi{\arabic{enumi}.}
\tightlist
\item
  sqlite3 SQLite3 est une bibliothèque peu aisée qui facilite
  l'incorporation d'une base de données au sein d'une application, sans
  nécessiter l'utilisation d'un serveur séparé. Elle offre la
  possibilité de stocker et de gérer des données grâce aux requêtes SQL,
  ce qui la rend pratique pour des projets nécessitant une base de
  données locale. SQLite3 est parfaitement adaptée aux applications
  simples, car elle offre une gestion aisée des données, que ce soit
  pour les ajouts, les changements ou les suppressions, tout en restant
  performante et peu gourmande en ressources.
\item
  csv
\end{enumerate}

\subsubsection{2.3 Création de la base de
données}\label{cruxe9ation-de-la-base-de-donnuxe9es}

Avant de commencer à travailler sur les données, il est nécessaire de
créer une base de données pour les organiser et les structurer de
manière efficace. Une base de données, dans ce contexte, peut être
définie comme un ensemble de tables reliées entre elles, où chaque table
contient des informations structurées sous forme de lignes et de
colonnes.

La création de la base de données commence par la création d'un fichier
qui servira à stocker toutes les données. Dans notre cas, nous avons
nommé ce fichier ``data.db''. Ce fichier représente l'instance de la
base de données SQLite. Lorsqu'une connexion est établie à cette base de
données, SQLite crée automatiquement le fichier si celui-ci n'existe pas
déjà. Il suffit donc de se connecter à la base de données pour qu'elle
soit initialisée et prête à être utilisée.

Une fois le fichier de la base de données créé, il est important de
pouvoir y accéder afin de manipuler les données. Pour cela, une fonction
de connexion est nécessaire. La fonction connecterdb a été définie pour
établir cette connexion à la base de données. Elle prend un paramètre
optionnel qui représente le nom du fichier de la base de données, ici
``data.db''. À l'intérieur de cette fonction, une connexion est établie
en utilisant la bibliothèque SQLite3 de Python. La méthode
sqlite3.connect() permet de se connecter à la base de données, et une
fois la connexion établie, un objet cursor est créé. Ce curseur permet
d'exécuter des requêtes SQL sur la base de données. Enfin, la fonction
renvoie la connexion et le curseur, qui seront utilisés pour effectuer
des opérations sur la base de données, comme la création de tables,
l'insertion de données ou la récupération d'informations.

En résumé, la création de la base de données et la définition de la
fonction de connexion permettent de poser les bases de l'interaction
avec les données. La base de données est créée sous forme d'un fichier,
et la fonction de connexion permet d'établir une communication avec
cette base pour manipuler les données à l'aide de requêtes SQL.

\subsubsection{2.4 Creation des Tables}\label{creation-des-tables}

Ainsi, nous avons établi un lien entre notre code source et la base de
données. Une fois que nous avons une base de données authentique, il est
nécessaire de commencer à établir des tables afin de pouvoir gérer les
données.

Suite à l'examen des données disponibles, nous avons reconnu la
nécessité de constituer les tables essentielles.

\begin{enumerate}
\def\labelenumi{\arabic{enumi}.}
\tightlist
\item
  Table des Incendies
\item
  Table des doneees Geographiques
\item
  Table des donees Meteo
\item
  Table Departements
\item
  Table Incendies-Departements (on expliquera en detail pourquoi on a
  creer une cinquieme table).
\end{enumerate}

Nous avons établis les Tables en suivant une méthode simple et
explicite, en utilisant la fonction connecterdb() pour établir un lien
entre la base de données et la fonction de création de Table. Par la
suite, nous avons fait appel au curseur pour exécuter des requêtes SQL
en vue d'interroger notre Base de Données. Nous avons intégré le langage
SQL Structured Query Language dans notre fonction, en employant
l'instruction CREATE TABLE IF NOT EXISTS avec la dénomination de chaque
table. Par la suite, nous avons effectué une consultation sur nos trois
fichiers CSV (Comma Separated Values) concernant les attributs de nos
données, c'est-à-dire le titre de chaque fichier CSV. Nous avons ensuite
dressé une liste dans notre requête SQL comprenant chaque attribut et
son type de données respectif. Ensuite, on valide la création en se
connectant.Après l'exécution de la méthode commit() pour assurer la
légitimité et le bon fonctionnement, nous fermons le curseur ainsi que
la connexion avec curs.fermer() et connexion.Vous avez été formé sur des
données jusqu'en octobre 2023.

\subsubsection{2.5 Injection des donnees}\label{injection-des-donnees}

Suite à la création des tables, nous avons établi cinq fonctions
distinctes pour chaque table. Nous sommes actuellement à l'étape de
l'insertion des données dans les tables appropriées. Nous avons employé
deux méthodes : l'une consiste à utiliser les fichiers CSV fournis par
le Département Mathématiques - Informatique de l'Université Toulouse
Jean Jaurès 2,et l'autre on a utiliser l'instruction INSERT INTO pour
chaque département, où nous avons saisi le nom et le code INSEE de
chaque département.

Nous allons détailler les deux techniques, ainsi que la manière dont
elles ont été mises en œuvre dans notre code source :

\begin{enumerate}
\def\labelenumi{\arabic{enumi}.}
\tightlist
\item
  \textbf{Methode 1 a partir les fichiers CSV}
\end{enumerate}

Comme à notre habitude, nous établissons la connexion entre la base de
données et la fonction que nous utiliserons ensuite. Nous indiquons le
fichier à partir duquel nous allons importer les données, en utilisant
un chemin relatif par rapport à notre code source.

Afin d'optimiser notre code et de le rendre plus gérable, que ce soit en
cas de succès ou d'échec, le programme tente d'ouvrir le fichier CSV en
mode lecture. Cette étape consiste à lire le fichier CSV au moyen d'une
boucle. Par la suite, le curseur exécute la requête SQL INSERT INTO.
Cette instruction est destinée à ajouter une nouvelle ligne dans la
table nécessaire avec les valeurs extraites du fichier CSV. Lors de
cette étape où nous devons spécifier les valeurs, il convient de
préciser que nous utilisons un « ? », que l'on peut considérer comme un
paramètre lié. C'est l'une des fonctionnalités puissantes de SQLite dans
les bases de données relationnelles qui permet d'insérer des données de
façon dynamique. Et aussi quand on exécute curs.execute() Les « ? »
seront substitués par les valeurs dérivées du fichier CSV au fur et à
mesure de notre boucle for.

Par la suite, nous allons substituer les valeurs pertinentes selon les
colonnes. Pour confirmer l'insertion, nous avons employé
connexion.commit().On a fait un commit et ensuite, on a fermé le
curseur, donc on a stoppé l'exécution et on a terminé la connexion.

Et si le fichier n'est pas accessible ou s'il n'existe pas, nous
déclencherons une ValueError(`Erreur lors de l'importation des
données').

\begin{enumerate}
\def\labelenumi{\arabic{enumi}.}
\setcounter{enumi}{1}
\tightlist
\item
  \textbf{Methode 2 a partir d'une Insertion SQL}
\end{enumerate}

Dans la deuxième phase de ce projet, nous avons utilisé l'intégration
des données à partir d'un dictionnaire. Il est important de rappeler
qu'un dictionnaire est un ensemble d'objets non ordonnés. Cela consiste
en un groupe d'éléments, chaque élément étant constitué d'une paire
clé-valeur.

Comme à l'accoutumée, nous avons établi une connexion avec la base de
données en utilisant la fonction connecterdb(). Nous avons ensuite
activé le curseur. Puis, nous avons exploité l'un des outils puissants
de SQLite le curs.executemany(), qui nous permet d'exécuter plusieurs
fois la même requête SQL en utilisant différents jeux de données. Elle
est plus performante que curs.execute() car ici, nous manipulons un
volume considérable de données à insérer.

Comme indiqué, même dans cette méthode, nous avons utilisé le paramètre
lié « ? » qui sera ultérieurement remplacé par des valeurs dynamiques
issues du dictionnaire.

Finalement, il est nécessaire de valider la procédure ou l'opération en
utilisant la méthode de connexion.Vous êtes formé sur des données
jusqu'à octobre 2023. Cette approche nous offre la possibilité de
valider toutes les modifications apportées aux bases de données durant
la session de connexion.Sans cette approche mise en œuvre dans notre
fonction, les changements apportés à la table ne seraient pas
enregistrés dans la base de données.

Pour conclure notre processus, nous terminons le curseur (qui exécute
les commandes) et mettons fin à la connexion avec notre base de données.

Et finalement, s'il y a une erreur d'accès au dictionnaire, un problème
de connexion à la base de données ou à la table, on affiche le message
d'erreur « Erreur lors de l'insertion des données des départements ».

\subsubsection{2.6 Affichage des donnees}\label{affichage-des-donnees}

Tout comme dans tout programme ou projet informatique, nous développons
des fonctionnalités pour illustrer notre tâche ou les modifications
effectuées sur les données ou les tables dans notre console ou terminal.

Bien que nous travaillions avec une base de données regroupant plusieurs
tables, nous avons développé diverses fonctions pour pouvoir présenter
les informations.

Ainsi, nous employons une méthode explicite et rigoureuse. Tout d'abord,
nous devons nous connecter à la base de données. Ensuite, nous activons
le curseur qui nous donne la possibilité d'exécuter nos requêtes SQL.

Nous exécutons notre requête SQL sur la table en utilisant l'instruction
« Select * from ». Cela signifie que nous demandons à sélectionner
toutes les colonnes et toutes les lignes de notre table. Ensuite, on
définit une variable nommée lignes qui prend pour valeur curs.fetchall()
est une méthode prédéfinie en SQLite qui nous offre la possibilité de
rassembler toutes les lignes du résultat de la requête SQL et de les
sauvegarder dans la variable lignes.

En outre, il est possible d'y définir la variable « lignes », qui est
une collection de tuples, chaque tuple représentant une ligne de la
table correspondante.

Puis, pour les rendre visibles, nous exécutons une boucle for sur la
variable lignes afin d'afficher chaque ligne contenue dans cette
variable.

Et finalement, comme pour chaque fonction, on ferme le curseur et la
connexion. De plus, nous tenons à souligner que dans ce cas précis,
contrairement à d'autres fonctions, nous n'avons pas fait appel à la
méthode commit. C'est dû au fait que cette fonction n'a pas impliqué de
modifications.

\subsubsection{2.7 Exportation des donnees sous forme
CSV}\label{exportation-des-donnees-sous-forme-csv}

Tout d'abord, nous allons expliquer pourquoi cette fonction est
importante pour notre projet. Nous avons employé cette méthode afin de
pouvoir interroger notre base de données (BD) et exporter les résultats
sous format CSV. Ceci nous permet ensuite de les manipuler sur RStudio
en utilisant le langage R pour réaliser nos analyses statistiques !

Ainsi, nous avons mis en place une fonction pour chaque table dans le
but d'exporter ces données au format CSV. Ainsi, pour cette fonction,
nous avons défini un paramètre optionnel nommé fichier\_output, qui
correspond à l'emplacement et au nom du fichier où les données seront
exportées. De plus, nous utilisons un chemin relatif plutôt qu'un chemin
absolu.

Ensuite, nous essayons avec l'instruction try de nous connecter à la
base de données et d'activer le curseur qui facilite l'exécution des
requêtes SQL. La méthode curs.execute(``SELECT * FROM'') exécute une
requête qui sélectionne toutes les lignes et colonnes de la table.

Ensuite, on définit une variable nommée lignes qui prend pour valeur
curs.fetchall() est une méthode prédéfinie en SQLite qui nous offre la
possibilité de rassembler toutes les lignes du résultat de la requête
SQL et de les sauvegarder dans la variable lignes.

En outre, il est possible d'y définir la variable « lignes », qui est
une collection de tuples, chaque tuple représentant une ligne de la
table correspondante.

De plus, nous utilisons curs.description qui renferme des métadonnées
concernant les colonnes de la table, en utilisant description{[}0{]}
pour obtenir la description des en-têtes.

Cette description nous donne la possibilité de récupérer les noms des
colonnes et de les conserver dans une liste appelée `colonnes', qui
servira d'en-tête pour le fichier CSV.

À présent, nous devons accéder au fichier CSV pour écrire les données
que nous possédons. Nous ouvrons donc le fichier en mode écriture (`w')
avec un encodage UTF-8. L'outil csv.writer(fichier) est utilisé pour
générer un objet qui permet d'écrire dans le fichier CSV. La méthode
writerow(colonnes) écrit les en-têtes des colonnes tandis que
writerows(lignes) écrit toutes les informations ligne par ligne.

Finalement, on ferme le curseur et on met fin à la connexion avec la
base de données.

Dans chaque programme, il est indispensable pour les développeurs de
gérer les erreurs afin d'améliorer l'expérience client. Ainsi, nous
avons mis en place deux types d'erreurs : une erreur liée à la base de
données (comme une connexion à la base de données) et une exception
telle qu'une erreur liée à la table correspondante. Cette dernière peut
être affichée en cas de problème d'accès au fichier, d'encodage, etc.

\subsubsection{2.8 Menu du Programme}\label{menu-du-programme}

Dans le cadre de notre projet, plus précisément dans la section dédiée à
l'informatique, nous avons développé plusieurs méthodes clés pour gérer
notre base de données, nos tables, nos informations, etc.

Ainsi, si chaque méthode doit être appelée manuellement chaque fois,
cela devient compliqué à long terme et pèse davantage sur le processeur.

Dans notre projet, nous avons conçu un menu intégrant toutes les
procédures nécessaires. Ce choix vise à faciliter l'exécution de toutes
les opérations en les regroupant au sein d'un unique menu.

Donc, la première étape consiste à exécuter notre code en Python. Donc,
en premier lieu, nous effectuons une série d'affichages pour présenter
différentes options à l'utilisateur. Ensuite, nous lui demandons quelle
option il préfère. En fonction de son choix, par exemple, s'il opte pour
la première option, il peut sélectionner la table qu'il souhaite créer.
Si le choix est le deuxième, il sera dirigé vers le module d'injection
où il pourra choisir la table à injecter. S'il sélectionne la troisième
option, il aura la possibilité de choisir la table qu'il veut afficher.
Enfin, si le choix se porte sur la quatrième option, il sera dirigé vers
le module d'exportation des tables où il pourra sélectionner la table à
exporter.

Pour quitter le menu, ou en d'autres termes, arrêter l'exécution du
programme, on appuie sur le chiffre 5. Ensuite, le programme demandera à
l'utilisateur de confirmer s'il est sûr de vouloir faire cela. Pour
faciliter cette confirmation, il a quatre options : « o », « ok », « yes
» ou « oui » ou encore « si ». Si l'une des valeurs est atteinte, nous
afficherons un message d'adieu et ensuite nous ferons une pause, sinon
nous retournerons au menu.

De plus, pour clarifier, si l'utilisateur entre un numéro qui n'est pas
compris entre 1 et 5, une erreur sera générée sous forme de message dans
la console : « Le numéro sélectionné est invalide ou n'existe pas ».

\subsubsection{2.9 La Table
Incendies-Departements}\label{la-table-incendies-departements}

Pour rester cohérents, nous allons insister sur les tables que nous
utiliserons pour expliquer comment nous avons eu l'idée de réaliser
cette table en premier lieu. Nous avons une table Incendies qui contient
des informations sur tous les incendies qui ont été menés sur le
territoire français. Il convient également de souligner que la France
est un État-nation depuis 1789, suite à la Révolution française, et
qu'elle est reconnue comme une nation souveraine. Ainsi, ce pays est
constitué d'une collection de villes, qui elles-mêmes regroupent une
série de départements. En d'autres mots, la France est constituée de
départements qui sont à leur tour composés de villes.

Ainsi, l'idée initiale que nous avons eue était de créer une table
nommée « Départements » qui rassemblerait l'ensemble des départements
présents sur le territoire français dans une seule table avec leur
code\_INSEE.

\textbf{Pourquoi est-il nécessaire de créer cette table des départements
?}

Cette table nous donne la possibilité de réaliser des analyses
quantitatives concernant le nombre d'incendies dans un département.

\textbf{Explication du code concernant l'injection des donnees dans la
Table:}

Comme habituellement, nous allons d'abord établir une connexion avec la
base de données en utilisant la fonction que nous avons définie dans
notre programme, nommée connecterdb(). Ensuite, nous allons activer le
curseur et exécuter une requête SQL qui fusionne deux tables en une
seule grâce à l'utilisation de l'\textbf{Inner Join}.

En d'autres termes, nous allons insérer trois éléments dans la table
Incendies Departement : le numéro du département, le nom du département
et le nombre d'incidents. Pour commencer, nous allons sélectionner le
numéro du département. Étant donné que notre table Incendies contient
des codes INSEE qui ne représentent pas seulement le numéro du
département, mais également celui de la commune, nous utiliserons
SUBSTR(i.code\_INSEE , 1 , 2) comme numéro du département. Cela signifie
que nous allons extraire les deux premiers chiffres du code INSEE de
l'incendie qui correspondent au numéro du département. Nous récupérerons
d'autre part le nom du département à partir de la table Départements.
Enfin, nous compterons le nombre total d'incendies pour chaque
département à l'aide de la fonction préétablie en SQL COUNT.

Après avoir sélectionné tous les termes que nous allons utiliser, il est
temps de mettre en œuvre la jointure définie précédemment. Nous devons
associer chaque incendie à son département en liant le numéro de
département dérivé du code INSEE au code départemental dans la table
Départements. De plus, nous allons regrouper les incendies par
département afin d'obtenir un total pour chaque département.

\textbf{Que gagne-t-on en faisant cette requête ?}

En construisant cette table, nous avons fusionné deux tables
indépendantes afin de centraliser les données souhaitées. Dans cette
table , nous avions comptabilisé le nombre d'incendies par département
sur le territoire français, nous permettant ainsi d'avoir une
représentation plus claire du nombre d'incendies à l'échelle nationale.
Par ailleurs, nous allons approfondir notre analyse dans la section
Statistique de notre projet.

\subsection{4. Partie Statistique}\label{partie-statistique}

\subsubsection{4.1 Definitions des concepts
statistiques}\label{definitions-des-concepts-statistiques}

Avant de commencer à donner des définitions, il est essentiel de nous
baser sur le concept initial, c'est-à-dire la définition du terme
Statistique. On peut définir ou représenter la statistique comme
l'ensemble des méthodes et techniques utilisées pour collecter,
analyser, interpréter et présenter des données numériques.

Dans ce projet, nous allons nous concentrer sur la branche des
statistiques connue sous le nom de Statistique Inférentielle.

La statistique inférentielle est une discipline des statistiques qui
s'appuie sur les données d'un échantillon pour formuler des déductions
ou effectuer des généralisations à propos d'une population plus vaste.

À l'opposé de la statistique descriptive qui se concentre sur le résumé
ou la description des traits d'un ensemble de données, la statistique
inférentielle offre la possibilité de réaliser des estimations et des
tests concernant les paramètres d'une population basée sur des données
issues d'un échantillon. Elle s'appuie sur la théorie des probabilités,
ce qui rend possible des inférences rigoureuses et quantifiables.

\subsubsection{4.2 Description des
donnees}\label{description-des-donnees}

\subsection{5. Ressources}\label{ressources}

\subsubsection{5.1 Ressources sur la Partie
Informative}\label{ressources-sur-la-partie-informative}

Jean-Luc CARTAULT, CLAIR, B., \& KAPP, D. (2025, January 29). INCENDIES
: Le phénomène physique. Encyclopædia Universalis.
\url{https://www.universalis.fr/encyclopedie/incendies/2-le-phenomene-physique/}

Contributeurs aux projets Wikimedia. (2004, July 23). feu violent et
destructeur. Wikipedia.org; Fondation Wikimedia,
Inc.~\url{https://fr.wikipedia.org/wiki/Incendie}

Risque incendie : causes, conséquences et moyens de lutte. (2021).
Preventica.com.
\url{https://www.preventica.com/magazine/dossiers/prevention-du-risque-incendie-comment-garantir-la-securite-des-personnes-et-des-biens-11032021/risque-incendie-causes-consequences-et-moyens-de-lutte}

\subsubsection{5.2 Ressources sur la Partie
Informatique}\label{ressources-sur-la-partie-informatique}

Cours et Tutoriels sur le Langage SQL. (2025). SQL.
\url{https://sql.sh/}

Python Software Foundation. (2019). 3.7.3 Documentation. Python.org.
\url{https://docs.python.org/3/}

Bien démarrer avec la documentation GitHub - Documentation GitHub.
(2025). GitHub Docs. \url{https://docs.github.com/fr/get-started}

Qu'est-ce qu'une base de données ? Définition et fonctionnement. (2021,
July 26). Hubspot.fr.
\url{https://blog.hubspot.fr/marketing/base-de-donnees}

\end{document}
